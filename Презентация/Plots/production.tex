% Динамика производства
\begin{tikzpicture}
    \begin{axis}[
        ybar stacked,
        font=\footnotesize,
        width=\textwidth, % Масштабирование графика до ширины страницы
        height=0.8\textheight,
        xmin=2010, xmax=2022,
        ylabel={тонн}, % Подпись оси Y
        ylabel style={
            at={(yticklabel cs:1,0)},
            anchor=south east
            },
        grid=major, % Отображение основной сетки
        grid style={solid, gray!30}, % Стиль сетки
        legend style={
          at={(0.5,-0.13)},
          legend columns=-1,
          anchor=north,
          draw=none}, % Размещение легенды под графиком
        x tick label style={rotate=90,anchor=east}, % Поворот меток по оси X
        xticklabel style={/pgf/number format/1000 sep=}, % Убираем разделитель тысяч
        yticklabel style={/pgf/number format/1000 sep={\ }
        },
        xtick=data, % Используем данные для меток на оси X
      ]
      \draw [pattern=north west lines, pattern color=gray!50]
      (rel axis cs:0,0) rectangle (rel axis cs:1,1);
      \addplot[fill=blue] table[
        x=Date,
        y=Australia,
        col sep=tab, % Указываем, что разделитель столбцов — табуляция
      ] {Data/Production/production.csv};
      \addlegendentry{Австралия}

      \addplot[fill=red] table[
        x=Date,
        y=Australia,
        col sep=tab, % Указываем, что разделитель столбцов — табуляция
      ] {Data/Production/production.csv};
      \addlegendentry{Бахрейн}
    \end{axis}
  \end{tikzpicture}

  %ymax=2800, ymin=1400,
  %ytick distance=200,
  %bar width=0.5cm,