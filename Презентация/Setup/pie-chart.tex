\newcommand{\pieslice}[6][black!10]{
%%% Usage: \pieslice[color]{total}{start angle}{end angle}{data value}{label}
  % calculate start and end points of arc
  \pgfmathparse{#3/#2*360}
  \let\a\pgfmathresult
  \pgfmathparse{#4/#2*360}
  \let\b\pgfmathresult

  % calculate mid angle of arc
  \pgfmathparse{0.5*\a+0.5*\b}
  \let\midangle\pgfmathresult

  % draw slice
  \draw[fill=#1] (0,0) -- (\a:1) arc (\a:\b:1) -- cycle;

  % outer label
  \node[label=\midangle:{\tiny#6}] at (\midangle:1) {};

  % inner label
  \pgfmathparse{min((\b-\a-10)/110*(-0.3),0)}
  \let\temp\pgfmathresult
  \pgfmathparse{max(\temp,-0.5) + 0.8}
  \let\innerpos\pgfmathresult
  \pgfmathparse{(\b-\a)/3.6} % convert slice size to percentage
  \let\percentage\pgfmathresult
  \node at (\midangle:\innerpos) {\tiny\pgfmathprintnumber[fixed,precision=1]{\percentage}\%};
}

\newcommand{\pie}[2][{{"black!10"}}]{
%%% Usage: \pie[{colour palette array}]{{label/value array}}
  % init colour palette
  \pgfmathparse{dim(#1)} % find N of array
  \let\paletteDim\pgfmathresult
  \newcounter{colourIndex}

  % get total for dividing pie into sectors
  \newcounter{total}
  \foreach \val/\name in #2 {
    \addtocounter{total}{\val}
  }

  \newcounter{a}
  \newcounter{b}
  \foreach \val/\name in #2 {
    \setcounter{a}{\value{b}}
    \addtocounter{b}{\val}

    % get colour from palette
    \pgfmathparse{#1[\thecolourIndex]}
    \let\colour\pgfmathresult

    \pieslice[\colour]{\thetotal}{\thea}{\theb}{\val}{\name}

    % increment colour palette
    \stepcounter{colourIndex}
    \ifnum \thecolourIndex=\paletteDim \setcounter{colourIndex}{0}\fi
  }
}

\def\palette{{"gold","blue","darkblue","red","yellow",
    "Color6","Color7","Color8","Color9", "Color10", "white"}}